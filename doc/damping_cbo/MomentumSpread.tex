\section{Momentum spread}
The energy dependence of the betatron tunes and the revolution frequency, contribute to the evolution 
and decoherence of betatron motion due to the finite energy spread in the beam.
In a cyclotron with electrostatic focusing
distributed uniformly around the ring,
the tune depends on the focusing index $n$ according to
\begin{eqnarray*}
Q_x &=& \sqrt{1-n}\\
Q_y &=& \sqrt{n}
\end{eqnarray*}
where $$n =\left(\frac{r}{v_s B}\right)\partder{E_r}{r}$$
$r$ is the radius of curvature of the on momentum muon in magnetic field $B$ and $v_s$ is the azimuthal
velocity, that is $r = \frac{\gamma m v_s}{q B} = \frac{p}{qB}$.
The dependence of betatron tune on energy, (chromaticity) follows from
$$n(\delta) = \frac{p(1+\delta)}{qv_s B^2}\partder{E_r}{r}$$ so
that
\begin{eqnarray*}
\partder{Q_x}{\delta} &=&-\frac{1}{2}\frac{1}{ \sqrt{1-n}}\partder{n}{p} = -\frac{n}{2\sqrt{1-n}} = \frac{Q_y^2}{2Q_x} =-0.103\\
\partder{Q_y}{\delta} &=& \frac{n}{2\sqrt{n}}=\onehalf Q_y = 0.215 
\end{eqnarray*}
evaluated for $n=0.185$.
An important assumption in the above is that the electrostatic quadrupoles extend continuously around the circumference of the ring
which is not the case in the g-2 ring. To understand the implications for chromaticity it is useful to 
distiguish three contributions; energy dependence of quadrupole focal length, energy dependence of pathlength
and effect of quad curvature.

The horizontal and vertical tunes are written in terms $\beta$-function
\begin{eqnarray}
Q_{h/v} = \frac{1}{2\pi}\oint \frac{1}{\beta_{h/v}} ds\nonumber\\
\end{eqnarray}
In a ring with constant $\beta$ ($\der{\beta}{s}=0$), we have that $K=1/\beta^2$ and the tunes are
\begin{eqnarray}
Q_{h/v} = \frac{1}{2\pi}\oint  \sqrt{K_{x/y}} ds \nonumber\\
Q_{h/v} =  \sqrt{K_{x/y}} R \label{eq:tunechrom}
\end{eqnarray}
where 
\begin{eqnarray*}
K_x &=& \frac{1}{\rho^2} - \frac{q}{mv^2}\partder{E}{r} = \frac{1}{\rho^2}-J\\
K_y &=& \frac{q}{mv^2}\partder{E}{y}\\
\end{eqnarray*}
where for convenience we define $J \equiv \frac{q}{mv^2}\partder{E}{r}=n$ and note that $Q_y = \sqrt{J}$.

The three distinct contributions to the chromaticity are enumerated.
\begin{enumerate} 
\item
The effective gradient (K) decreases with
energy, increasing horizontal and decreasing the vertical tune.
\begin{eqnarray*}
\partder{K^{1/2}_x}{\delta} &=& \onehalf K^{-1/2} \partder{K}{\delta} = -\frac{(\rho^{-2} -\onehalf J )}{\sqrt{K}} = -\sqrt{K} - \frac{J}{\sqrt{K}} = -Q_x - \frac{Q_y^2}{2Q_x}\\ 
\partder{K^{1/2}_y}{\delta} &=&  -\onehalf\sqrt{K} = -\onehalf Q_y
\end{eqnarray*} 
where we have assumed rectangular coordinates so that $\partder{E_r}{r}=\partder{E_x}{x}=-\partder{E_y}{y}$
\item The path length increases with energy, $\Delta P = 2\pi\eta \delta$. 
If the inner and outer quad plates
have equal angular length, then a longer path corresponds to longer quads and more focusing.
$$\partder{Q_{h/v}}{\delta} =  \sqrt{K_{x/y}}  \eta = \frac{\eta}{R}Q_{h,v}$$
where $R$ is the magic radius. If the quad plates have equal linear length (rather than equal angular length),
then there is no pathlength dependent focusing. The effect of pathlength will depend on the details of the fringe field at the ends of the quads.
\item The sextupole component of the quad fields couples to the tune via the dispersion.
If the quad plates have no curvature, then the quadrupole symmetry precludes a sextupole moment.
But there is curvature in the g-2 quads, and solutions to Laplace's equation in cylindrical coordinates
are guaranteed a sextupole component.
In cartesian coordinates the quadrupole potential $$V(x,y) = \onehalf k( x^2 - y^2)$$ gives 
$${\bf E} = -{\bf\nabla} V = -kx{\bf\hat x} + ky{\bf\hat y}$$ and $\partder{E_x}{x}+\partder{E_y}{y}=0$.
In cylindrical coordinates, a solution (but not a unique solution) to the Laplace equation with lowest order term linear in displacement, all of the nonlinearity
in the radial coordinate, and a strictly linear vertical dependence is
$$V = k\left(\onehalf(r^2-1) - r_0\ln \frac{r}{r_0} - y^2\right)$$
and 
\begin{eqnarray*}
{\bf E} &=& \onehalf k\left((r-\frac{r_0^2}{r}){\bf \hat r} -2y{\bf \hat y}\right)\\
\end{eqnarray*}
With the substitution $r=r_0+x$, where $r_0$ is the magic radius,
\begin{eqnarray*}
{\bf E} &=& \onehalf k\left((r_0+x-\frac{r_0^2}{r_0}[1-\frac{x}{r_0}+\onehalf \left(\frac{x}{r_0}\right)^2+\ldots]){\bf \hat r} -2y{\bf \hat y}\right)\\
 &\sim&  k\left(x- \frac{x^2}{2r_0}+\ldots){\bf \hat r} -y{\bf \hat y}\right)\\
\end{eqnarray*}
The closed orbit for an off energy particle is shifted to $x\rightarrow \eta\delta + x$ and the radial component of the field
for an off energy muon becomes
\begin{eqnarray*}
E_r &=& k\left(x+\eta\delta-\frac{1}{2r_0}(x+\eta\delta)^2\right)\\
\partder{E_r}{x} &\rightarrow&  k(1-\frac{1}{r_0}\eta\delta)\\
\partder{\sqrt{K_x}}{\delta} &=& -\sqrt{K_x}\frac{\eta}{2r_0}
\end{eqnarray*}
The contribution to the chromaticity due to the quadratic component of the electric field is
$$\partder{Q_{h}}{\delta} = -\frac{\eta}{2r_0}\sqrt{K_x} = \frac{\eta}{2r_0}Q_x$$
\end{enumerate}
In summary, contributions to chromaticity include \begin{enumerate}
\item the energy dependence of the quad gradient and bend curvature
\item energy dependence of path length
\item  nonlinearity associated with the curvature.
\end{enumerate}
Assuming equal angular length of inner and outer plates and strictly linear vertical quad focusing (so that all of the nonlinearity due to curvature appears in the horizontal)
the sum of the contributions is
$$\partder{Q_{x}}{\delta} = -Q_{x}-\frac{Q_y^2}{2Q_x} +Q_x\frac{\eta}{r_0} -Q_x\frac{\eta}{2r_0}$$
$$ \partder{Q_y}{\delta}= -Q_{y}\left(\onehalf-\frac{\eta}{r_0}\right)$$
It turns out that for continuous focusing, $$\eta = \frac{1}{Kr_0} = \frac{r_0}{Q_x^2}$$
Then
$$\partder{Q_{x}}{\delta} = -Q_{x}-\frac{Q_y^2}{2Q_x} +\frac{1}{Q_x}(1-\onehalf)=\frac{-1}{2Q_x}+\frac{1}{Q_x}(1-\onehalf)=0$$
Evidently, if the inner and outer quad plates are equal angular length and if the quadratic correction to the quad field
is restricted to the radial direction so the vertical is linear, then the horizontal chromaticity is identically zero.
And the vertical can be written
$$ \partder{Q_v}{\delta}= -Q_{y}\left(\onehalf-\frac{1}{Q_x^2}\right)$$
If, on the other hand, angular length is different for inner and outer plates, and/or Laplace's equation in cylindrical coordinates
is satisfied by some nonlinearity in the vertical, rather than the horizontal, the chromatities will be very different.
In order to get the chromaticity right, we need a 3D map of the quad fields that includes end effects as well as curvature.


%We rewrite Equation ~\ref{eq:tunechrom} in terms of field index
%\begin{eqnarray*}
%Q_{h/v} &=& \frac{1}{4\pi}\oint \beta_{h/v} \frac{v_s B n}{pr} ds\\
%&\sim& \frac{1}{4\pi}\beta_{h/v} \frac{v_s B n}{pr} 2\pi r \\
%&\sim& \frac{1}{2}\beta_{h/v} \frac{v_s B n}{p} = -0.103/0.215
%\end{eqnarray*}


%Note that the horizontal tune decreases with energy while the vertical tune increases with energy.
%In the g-2 ring the quadrupoles extend over only $156^{\circ}$ of the circumference and
%a more careful calculation that accounts for the nonuniformity of the $\beta$-functions gives
%\begin{eqnarray*}
%Q^\prime_x&=& -0.14\\
%Q^\prime_y&=& 0.31
%\end{eqnarray*}

Finally, suppose that all of the particles in the initial distribution appear in the ring at the
same point in space and time but with a spread in energy. The particles will execute betatron
oscillations with a frequency that depends on the energy, namely $Q_x(\delta) = Q^\prime_x\delta$
and $Q_y(\delta) = Q_y^\prime \delta$, where $\delta$ is the fractional energy offset. 
The particles will circulate with cyclotron frequencies
$\frac{1}{\omega(\delta)} = \frac{1+\delta}{\omega_0}$.
The betatron frequency
\begin{eqnarray*}
\omega_\beta &=& Q\omega \ \ \ {\rm becomes} \\
\omega_\beta &\rightarrow& (Q+Q^\prime\Delta)\frac{\omega_0}{1+\Delta}\\  
%&\sim& (Q+Q^\prime\delta)\omega_0(1-\delta)\\
%&\sim& Q\omega_0 +\delta(Q^\prime-Q)\omega_0 
\end{eqnarray*}
%The average over all momenta is
%\begin{eqnarray*}
%\langle x \rangle &=& x_0\left(\cos(Q\omega_0 t)\langle\cos(Q^\prime-Q)\delta\omega_0 t\rangle
%-\sin Q\omega_0 t \langle \sin(Q^\prime-Q)\delta\omega_0 t\rangle\right)\\
% &=& x_0\left(\cos(Q\omega_0 t)\sum_{\delta_i}\cos(Q^\prime-Q)\delta_i\omega_0 t
%-\sin Q\omega_0 t\sum_{\delta_i} \sin(Q^\prime-Q)\delta\omega_0 t\right)\\
%\end{eqnarray*}
%For any energy distribution symmetric about zero, the ``$\sin$'' term vanishes. 

%\subsection{Flat energy distribution}
%If the distribution
%is flat over the range $\pm\delta_0$ then
%\begin{eqnarray}
%\sum_{\delta}\cos(Q^\prime-Q)\delta_i\omega_0 t &\rightarrow& \frac{1}{2\delta_0}\int_{-\delta_0}^{\delta_0}
%\cos (Q^\prime -Q)\delta_i\omega_0 t d\delta\nonumber\\ 
%&=& \frac{1}{\xi}
%\left(\sin(\xi \delta_0)+\sin(\xi\delta_0)\right)\nonumber\\
%&=& 2\frac{\sin\xi \delta_0}{2\delta_0\xi}\label{energy_width}
%\end{eqnarray}
%where $\xi = (Q^\prime-Q)\omega_0 t$. According to~\ref{energy_width},
%the average position of all of the particles is zero when
%\begin{eqnarray*}
%\xi\delta_0 &=& \pi \\
%\rightarrow t&=& \frac{\pi}{(Q^\prime-Q)\omega_0\delta_0}
%\end{eqnarray*}
%Let's compute the horizontal and vertical decoherence times in units of the revolution period.
%\begin{eqnarray*}
%N_x &=& \frac{\omega_0 t}{2\pi} = \frac{\pi}{(-0.14 - 0.908)2\pi(0.0012)}=909\\
%N_y &=& \frac{1}{2(0.31-0.43)0.0012} = 3571
%\end{eqnarray*}

%\subsection{Gaussian energy distribution}
%If the distribution of energies is Gaussian with width $\delta_0$ then
%$$x(t) = \delta\eta\cos\left((Q+(Q^\prime-Q)\delta)\omega_0 t\right)$$
%\begin{eqnarray*}
%\sum_{\delta}\cos(Q^\prime-Q)\delta_0\omega t&\rightarrow& \frac{1}{\delta_0\sqrt{2\pi}}\int_{-\infty}^\infty\cos\xi\delta {e^{-\frac{1}{2}\frac{\delta^2}{\delta_0^2}}}d\delta\\
%&=& \frac{1}{2\delta_0\sqrt{2\pi}}
%\int_{-\infty}^\infty\left(e^{i\xi\delta}+e^{-i\xi\delta}\right) 
%e^{-\frac{1}{2}\frac{\delta^2}{\delta_0^2}}d\delta\\
%&=& \frac{2}{2\delta_0\sqrt{2\pi}}e^{-\frac{1}{2}\xi^2\delta_0^2}(\sqrt{2\pi}\delta_0)\\
%&=& e^{-\frac{1}{2}\xi^2\delta_0^2}
%\end{eqnarray*}
