\section{Introduction}

The Fermilab E989 Muon g-2 experiment aims to measure the anomalous part \amu\ of the magnetic moment of the muon.
The muon acquires a magnetic moment 
\begin{equation} 
\vec{\mu}=g \frac{Q}{2m} \vec{s},
\end{equation} in the presence of an external magnetic field $B$, where $Q$ is the muon electric charge, $\vec{s}$ the muon spin vector 
and $g$  the gyromagnetic ratio. 
The anomalous part of the magnetic moment is defined via the deviation of the gyromagnetic ratio: $g = 2(1+a_\mu)$.

The Muon g-2 experiment relies on the storage of muons inside a weak focusing ring. 
A continuous C-shape dipole magnet occupies the entirety of the storage ring (44.7 m circumference). 
It provides the 1.45 T inward radial focusing to store the muons in the ring. 
The field intensity corresponds to storing muons with the so-called magic momentum of \mbox{3.09 GeV/c} onto the magic orbit (7.112 m radius).
The muons undergo a cyclotron motion in the ring with a revolution frequency of 149.1 ns.

The weak focusing ring does not provide vertical focusing which is essential to store the muons. 
The vertical focusing is provided by four electrostatic quadrupoles (ESQ) located around the ring.
In their rest frame, the muons see the electric field generated by the ESQs as a magnetic field.

The measurement of \amu~ is performed via the measurement of the intensity of the magnetic field in terms of the Larmor precession frequency of a free proton 
\begin{equation}
\hbar \omega_{p} = 2 \mu_{p} | \overrightarrow{B}|,
\end{equation}
and the intrinsic spin precession frequency of the muon \wa. 
The intrinsic muon spin precession frequency is obtained by subtracting the cyclotron frequency $\omega_C$ to the total spin precession frequency of the muon $\omega_S$:
\begin{equation}
\wa = \omega_S - \omega_C.
\end{equation}

Equation~(\ref{eq:amu}) shows the most general vectorial relation between \amu~ and \wa, $B$:
\begin{equation}
\label{eq:amu}
\vec \omega_{a}=\vec \omega_S -\vec\omega_C 
=  - \frac{Qe}{ m}
\left[ a_{\mu} \vec B -  a_{\mu}\left( {\gamma \over \gamma + 1}\right)
(\vec \beta \cdot \vec B)\vec \beta 
- \left( a_{\mu}- {1 \over \gamma^2 - 1} \right) 
{ {\vec \beta \times \vec E }\over c }\right]
\end{equation}

The term proportional to ${\vec \beta \times \vec E }$ in Eq.~(\ref{eq:amu}) corresponds to the electric field contribution to \wa.
One can see that if $a_{\mu} = {1 \over \gamma^2 - 1}$ the contribution disappears. 
This is the approach followed by the previous CERN and BNL E821 experiments. 
The magic momentum of 3.09 GeV/c allows the electric field contribution to vanish in first order.

The non-vanishing electric field contribution arises from the fact that the stored muon beam has a momentum spread of about 0.1\%.
This effect is non-negligible and needs to be taken care of. The estimated correction to \wa~ due to the electric field was estimated
by E821 to be $0.47 \pm 0.05$ ppm for the 2001 data set for the low n-value data set. The final E821 result for \amu~ has a 0.54 ppm precision, which
is of the order of the electric field correction.

The term proportional to $\vec \beta \cdot \vec B$ in Eq.~(\ref{eq:amu}) corresponds to the so-called pitch correction due to the muon velocity not
being purely contained in the horizontal plane. This is another correction that needs to be addressed.

This note presents an attempt at estimating the electric field correction using the Fourier analysis technique applied to the Fast Rotation signal. 
This technique was developed by E821 and detailed in~\cite{orlov}.
